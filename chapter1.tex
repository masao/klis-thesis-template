\chapter{はじめに}
\label{chap:introduction}

本稿は、知識情報・図書館学類の卒業論文の非公式のひな形を提供する。

以下では、卒業論文の書き方を題材として、\LaTeX による卒業論文執筆の例を示す。
卒業論文の書き方については別途文書を公開しているので、そちらも参考にすること\cite{takaku:howto}。

第1章は、最初の章であるため、研究の背景と目的をロジカルに説明する。
もっとも重要な視点は、「なぜあなたの研究が必要なのか?」という疑問に端的に答えることである\cite{sakai:2015:korekara}。

研究がなぜ必要とされているかを述べるにあたっては、社会的な側面、学術的な側面の両方から説明できることが望ましい。
さらに、それらを説明するにあたっては、できるだけ参照文献を用いながら説明を進めることが望ましい。

また、説明のロジックは、パラグラフ・ライティング (paragraph writing) 方式で、1) アウトラインを最初に考えて、2) 段落ごとの流れを確認、3) 各段落の内容を肉付けするといった方式をとると効率が良いと思われる。
例えば、高久が過去に執筆した論文「タスク種別とユーザ特性の違いがWeb情報探索行動に与える影響」(情報知識学会誌, 2010)\cite{takaku:2010:jsik} における第1章は、以下のようなアウトラインから構成されている:

\begin{enumerate}
    \item Webサーチエンジンの重要性
    \item Exploratory searchの導入
    \item これまでの研究法の視点と課題
    \item 研究の目的: 包括的な情報探索データ分析に基づく探索行動の精緻な理解
    \item 論文の構成
\end{enumerate}

上記にもある通り、おおよそ多くの研究は、社会的な意義の説明、学術的な意義の説明、これまでの研究の到達点とその課題の紹介、解決方法(提案手法)の紹介、研究の目的といった流れになり、各項目が1~2パラグラフから構成されるように作るとアウトラインと文章作成に有用と思われる。

また、研究全体を貫くような RQ (Research Question) があれば、それらに番号を振って、先に示しておくとよい。

\begin{itemize}
    \item RQ1: 提案手法を用いると卒業論文の執筆が効率的に進むか?
    \item RQ2: 提案手法を用いると卒業論文の質は上がるか?
    \item RQ3: 提案手法を用いると卒業論文の執筆が効率的に進むか
\end{itemize}
