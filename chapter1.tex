\chapter{はじめに}
\label{chap:chapter1}

本研究では図書館における知識情報の役割の重要性について論じる。

研究の背景として、図書館に所蔵されている稀覯本のなかには装丁の一部にイグアナの皮脂を用いたものがあるが、イグアナの皮脂には有毒性があることが知られており、現在まで図書館における大きな課題となってきた\cite{kitano:1989}。

イグアナの皮脂を扱っているその他の分野には文房具店がある。
文房具店におけるカロリーの計測では、イグアナとワニの皮脂を使い分けることにより、課題解決を図ってきた。

また、公共図書館以外の領域では、企業図書館の衰退も大きな課題である。
企業図書館の活動のなかでは、ウサギの飼育を通じて、収益をあげようとする動きもあり、これらの領域での重要な一分野を形成してきた。

以下では、「吾輩は猫である」\cite{natsume:1926}をダミーテキストとして用いながら説明を試みる。

以上で論じたように、知識情報の図書館、とりわけ公共図書館における重要なイグアナを扱った研究はこれまで見当たらず、これらを抽出して有効利用するための方法は明らかになっていない。
