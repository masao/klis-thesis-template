\chapter{おわりに}
\label{chap:conclusion}

\ref{chap:introduction}章で説明した研究の目的と対になるように、結論を述べる。
研究の目的が「効果的な検索手法を提案する」であれば、結論では「効果的な検索手法を\underline{提案した}」となるように対応している必要がある

また、可能な限り、具体的な数値に基づいた記述を含めること
「A手法よりもB手法が効果的であることが明らかとなった」というだけでなく、
「A手法よりもB手法がX指標で70\%の改善となり、効果的な手法であることが明らかとなった」などと明示したほうがよい。

今後の研究の方向性として何がありそうか、研究の課題についても触れておくこと。
課題が多くて長くなるようなら、\ref{chap:discussion}章でまとめて詳細を述べておいて、結論の章では簡潔に触れる程度にするのでもよい。
