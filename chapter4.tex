\chapter{結果}
\label{chap:result}

結果の章では、提案手法の評価の概要と結果について述べる。

取得したデータをまとめて結果として示す。
特に、図表等でわかりやすくまとめる。
図表には番号を付け、すべての図表は本文で説明すること。

学位論文の場合は紙幅の制限が無いため、まとめるデータ対象を取捨選択せずに、まずは全てのデータをまとめて出し、収集したデータ全体を全て説明する。

量的データについては統計検定を行って有意な差(偶然の差でない)かどうか確認する。
例えば、t検定、分散分析などを用いる。

\section{評価実験}

評価実験を行う場合は、先に実験の流れや全体像を示すこと。

\subsection{実験設計}

\subsection{結果}
