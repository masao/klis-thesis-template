\chapter{関連研究}
\label{chap:relatedwork}

2章では関連研究の整理を行う。
単純に自分の研究に先行する類似研究を2~3紹介すれば終わりというのではなく、
そもそも取り組んでいる研究領域がどの程度の範囲にまたがるのか、
使っている研究手法や対象データの側面から、どのような研究が行われているのかなど、
学位論文であればページ数が限られないため、かなり広範に述べることが期待される。


\section{知識情報の構造化}

これまでに知識情報を構造化する研究は、さまざまな角度で試みられてきた。


\section{図書館における保全手法}

図書館、とりわけ公共図書館においてイグアナを保存する研究としては、大矢によるものが嚆矢である\cite{ohya:2004}。


\section{知識の可視化手法}

知識情報を可視化する手法全般は、情報可視化の一領域として構成されてきた\cite{kato:2019}\cite{aoi:2020}。

