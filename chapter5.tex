\chapter{考察}
\label{chap:discussion}

考察の章では、\ref{chap:result}章で出した結果を解釈する。
逆に言えば、結果の章では解釈をしたりすることは考察の章にまわすこととして、ほとんど解釈せずに結果を表示することに注力してしまってよい。

また、データによっては、新たにクロス集計の形でまとめなおしたりしたデータを示しながら、なぜそのような結果となったのか、なにが要因として考えられるか、うまくいっている点とうまくいっていない点など、要素分解しながら整理して検討を加えること。
将来の自分または研究室の後輩などが読んだときに参考になるように記述する

\section{実験結果からわかったこと}

先に、\ref{chap:introduction}章において示したRQにしたがって、どのようなことが言えるかを端的に示す。

\section{失敗分析}

データの前処理や提案アルゴリズムの処理などにおいては、どのような失敗が生じているか、
可能な範囲で、ありうる類型、パターンを整理しながら、失敗とその精度などとの関係について触れておくとよい。

\section{研究の限界点}

研究の限界点についても節を立てて議論しておこう。
例えば、使ったデータに制限があって、一部しか使えていないとか、一部のジャンルに限定されているとかいう結果に与えている影響があれば、どのような限界が出ていると考えるか、どのような影響が考えられるかなどについて議論しておくこと。
